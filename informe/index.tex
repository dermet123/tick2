\documentclass{article}
\usepackage[utf8]{inputenc}
\usepackage[spanish]{babel}
\usepackage{amsmath}
\usepackage{graphicx}
\usepackage{geometry}
\usepackage{hyperref}
\usepackage{listings}
\usepackage{xcolor}
\usepackage{booktabs}

\geometry{a4paper, margin=1in}

% Configuración para listados de código
\lstset{
    language=PHP,
    basicstyle=\ttfamily\small,
    numbers=left,
    numberstyle=\tiny,
    frame=single,
    breaklines=true,
    breakatwhitespace=true,
    showstringspaces=false,
    keywordstyle=\color{blue},
    commentstyle=\color{gray},
    stringstyle=\color{purple}
}

\title{Documentación del Sistema Generador de Tickets}
\author{Equipo de Desarrollo}
\date{\today}

\begin{document}

\maketitle
\tableofcontents
\newpage

\section{Introducción}
\subsection{Propósito}
Este documento proporciona una documentación completa del sistema Generador\_Tickets, una solución integral para la gestión de usuarios y la generación automatizada de facturas electrónicas. El sistema está diseñado para ser fácil de usar, seguro y escalable.

\subsection{Alcance}
El sistema abarca las siguientes funcionalidades principales:
\begin{itemize}
    \item Gestión de usuarios (crear, leer, actualizar, eliminar)
    \item Generación de facturas en formato PDF
    \item Interfaz web amigable
    \item API RESTful para integraciones
\end{itemize}

\subsection{Características Principales}
\begin{itemize}
    \item Generación automática de facturas numeradas
    \item Almacenamiento seguro de datos
    \item Interfaz responsiva
    \item Fácil instalación y configuración
    \item Código fuente bien documentado
\end{itemize}

\section{Estructura del Proyecto}
El proyecto se organiza en los siguientes directorios y archivos principales:

\begin{itemize}
    \item \textbf{.htaccess}: Archivo de configuración para el servidor Apache.
    \item \textbf{api/}: Contiene los archivos PHP para la gestión de la API.
    \item \textbf{carpeta\_pdfs/}: Almacena los archivos PDF generados.
    \item \textbf{config/}: Contiene archivos de configuración, como la configuración de la base de datos.
    \item \textbf{facturas/}: Posiblemente almacena facturas.
    \item \textbf{informe/}: Contiene este informe.
    \item \textbf{lib/}: Contiene librerías externas, como TCPDF.
    \item \textbf{pdfs/}: Almacena archivos PDF.
    \item \textbf{public/}: Contiene archivos públicos como HTML, CSS y JavaScript.
    \item \textbf{generatePDF.php}: Archivo PHP para generar facturas en formato PDF.
    \item \textbf{index.php}: Página principal del sistema.
\end{itemize}

\section{Componentes Clave}

\subsection{index.php}
La página \texttt{index.php} es la página de inicio del sistema. Permite crear un usuario.

\subsection{generatePDF.php}
El archivo \texttt{generatePDF.php} es responsable de generar las facturas en formato PDF. Utiliza la librería TCPDF para crear los documentos PDF. Recibe el ID del usuario como parámetro GET y genera una factura con los datos del usuario y otros datos estáticos.

\subsection{config/database.php}
El archivo \texttt{config/database.php} contiene la configuración de la base de datos. Define las credenciales de acceso a la base de datos MySQL.

\section{Instalación y Configuración}
\subsection{Requisitos del Sistema}
\begin{itemize}
    \item Servidor web Apache 2.4 o superior
    \item PHP 7.4 o superior
    \item MySQL 5.7 o superior
    \item Extensión PHP PDO habilitada
    \item Extensión PHP GD para generación de imágenes en PDF
\end{itemize}

\subsection{Configuración de la Base de Datos}
La configuración de la base de datos se define en el archivo \texttt{config/database.php}. Los parámetros de configuración son:

\begin{itemize}
    \item \textbf{Host}: \texttt{localhost}
    \item \textbf{Base de datos}: \texttt{proyecto\_php}
    \item \textbf{Usuario}: \texttt{root}
    \item \textbf{Contraseña}: (vacía)
\end{itemize}

\subsection{Pasos de Instalación}
\begin{enumerate}
    \item Clonar el repositorio en el directorio web
    \item Configurar los permisos de escritura en las carpetas necesarias
    \item Importar el esquema de la base de datos
    \item Configurar el archivo \texttt{config/database.php} con las credenciales correctas
    \item Acceder a la aplicación a través del navegador web
\end{enumerate}

\section{Frontend}
El frontend del sistema se encuentra en el directorio \texttt{public/}. Utiliza HTML, CSS y JavaScript para la interfaz de usuario.

\subsection{public/index.php}
El archivo \texttt{public/index.php} contiene el formulario para la creación de usuarios y la lista de usuarios. Utiliza Tailwind CSS para el diseño.

\subsection{public/script.js}
El archivo \texttt{public/script.js} contiene la lógica JavaScript para el formulario de creación de usuarios, la lista de usuarios y la interacción con la API.

\subsection{public/style.css}
El archivo \texttt{public/style.css} contiene estilos CSS adicionales para el frontend.

\section{API Endpoints}
El directorio \texttt{api/} contiene los archivos PHP para la gestión de la API RESTful. A continuación se detallan los endpoints disponibles:

\subsection{Usuarios}
\begin{itemize}
    \item \textbf{POST /api/users.php}: Crea un nuevo usuario.
    \item \textbf{GET /api/list.php}: Obtiene la lista de usuarios.
    \item \textbf{PUT /api/edit.php}: Actualiza un usuario existente.
    \item \textbf{DELETE /api/delete.php}: Elimina un usuario.
\end{itemize}

\subsection{Ejemplo de Uso de la API}
A continuación se muestra un ejemplo de cómo consumir la API usando cURL:

\begin{lstlisting}[language=bash,caption=Ejemplo de creación de usuario]
curl -X POST http://localhost/api/users.php \
     -H "Content-Type: application/json" \
     -d '{"nombre":"Juan Perez", "email":"juan@ejemplo.com"}'
\end{lstlisting}

\section{Guía de Uso}
\subsection{Generación de Facturas}
Para generar una factura, sigue estos pasos:
\begin{enumerate}
    \item Inicia sesión en el sistema
    \item Navega a la sección de facturas
    \item Completa el formulario con los datos del cliente
    \item Agrega los productos o servicios
    \item Haz clic en "Generar Factura"
\end{enumerate}

\section{Solución de Problemas}
\subsection{Problemas Comunes}
\begin{description}
    \item[No se pueden generar PDFs] Verifica que la carpeta de salida tenga permisos de escritura.
    \item[Error de conexión a la base de datos] Revisa las credenciales en el archivo de configuración.
    \item[Problemas de permisos] Asegúrate de que el servidor web tenga permisos de escritura en las carpetas necesarias.
\end{description}

\section{Seguridad}
\begin{itemize}
    \item Todas las contraseñas se almacenan con hash seguro
    \item Se recomienda configurar HTTPS para mayor seguridad
    \item Validación de entrada en todos los formularios
    \item Protección contra inyección SQL mediante consultas preparadas
\end{itemize}

\section{Licencia}
Este software se distribuye bajo la licencia MIT. Para más detalles, consulta el archivo LICENSE incluido en el repositorio.

\section{Contacto}
Para soporte o consultas, por favor contacta a:\
\textbf{Email:} soporte@ejemplo.com\
\textbf{Sitio web:} https://www.ejemplo.com

\section{Contribuciones}
Las contribuciones son bienvenidas. Por favor, lee las guías de contribución antes de enviar un pull request.

\section{Historial de Versiones}
\begin{itemize}
    \item \textbf{v1.0.0} (2025-05-28) - Versión inicial
    \item \textbf{v1.0.1} (Próximamente) - Mejoras de rendimiento
\end{itemize}

\end{document}
